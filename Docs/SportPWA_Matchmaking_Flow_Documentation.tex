\documentclass[12pt]{article}
\usepackage[utf8]{inputenc}
\usepackage[bahasa]{babel}
\usepackage{geometry}
\usepackage{amsmath}
\usepackage{amsfonts}
\usepackage{graphicx}
\usepackage{tikz}
\usepackage{pgfplots}
\usepackage{float}
\usepackage{algorithm}
\usepackage{algorithmic}
\usepackage{listings}
\usepackage{xcolor}
\usepackage{hyperref}
\usepackage{subcaption}
\usepackage{fancyhdr}
\usepackage{titlesec}

\geometry{
    a4paper,
    margin=1in
}

\pagestyle{fancy}
\fancyhf{}
\fancyhead[L]{SportPWA - Panduan Matchmaking System}
\fancyhead[R]{\thepage}
\fancyfoot[C]{© 2025 SportPWA - Panduan Praktis}

\usetikzlibrary{shapes,arrows,positioning,calc}

\definecolor{primaryblue}{RGB}{59, 130, 246}
\definecolor{secondarygreen}{RGB}{34, 197, 94}
\definecolor{warningorange}{RGB}{251, 146, 60}
\definecolor{dangerred}{RGB}{239, 68, 68}
\definecolor{codegray}{RGB}{248, 249, 250}

\lstset{
    basicstyle=\ttfamily\footnotesize,
    breaklines=true,
    frame=single,
    numbers=left,
    numberstyle=\tiny,
    showstringspaces=false,
    commentstyle=\color{gray},
    keywordstyle=\color{primaryblue},
    stringstyle=\color{secondarygreen},
    backgroundcolor=\color{codegray},
    captionpos=b
}

\title{
    \vspace{-2cm}
    {\Huge\textbf{SportPWA Matchmaking System}}\\
    \vspace{0.5cm}
    {\Large Panduan Lengkap dan Mudah Dipahami}\\
    \vspace{0.3cm}
    {\large Dari Join Event sampai Match Selesai}\\
    \vspace{0.3cm}
    {\large Version 2.0 - Januari 2025}
}

\author{
    SportPWA Development Team\\
    \texttt{team@sportpwa.com}
}

\date{\today}

\begin{document}

\maketitle

\tableofcontents
\newpage

\section{Pengantar - Apa itu SportPWA?}

SportPWA adalah aplikasi olahraga yang membantu pemain menemukan lawan bertanding yang seimbang. Bayangkan kamu ingin main badminton tapi bingung cari lawan yang levelnya sama - nah, SportPWA ini yang akan bantu carikan!

\subsection{Kenapa Perlu Sistem Matchmaking?}
\begin{itemize}
    \item \textbf{Fair Play}: Supaya pertandingan nggak ada yang menang telak
    \item \textbf{Credit Score}: Mengatur perilaku pemain biar nggak sering batal-batal
    \item \textbf{Queue yang Rapi}: Antrian yang tertib, nggak rebutan court
    \item \textbf{Rating System}: Tracking kemampuan pemain dari waktu ke waktu
\end{itemize}

\subsection{Komponen Utama}
\begin{itemize}
    \item \textbf{Event Management}: Daftar event, join, check-in
    \item \textbf{Fair Matchmaking}: Carikan lawan yang seimbang
    \item \textbf{Credit Score}: Poin perilaku (0-100)
    \item \textbf{MMR System}: Rating skill pemain per olahraga
    \item \textbf{Court Management}: Atur lapangan dan antrian
\end{itemize}

\section{Perjalanan Player: Dari Daftar sampai Main}

\subsection{Gambaran Umum Perjalanan Player}

Ini dia step-by-step yang dialami player di SportPWA:

\begin{figure}[H]
\centering
\begin{tikzpicture}[
    node distance=1.8cm,
    start/.style={rectangle, rounded corners, minimum width=3cm, minimum height=1cm, text centered, draw=primaryblue, fill=primaryblue!20},
    process/.style={rectangle, minimum width=3cm, minimum height=1cm, text centered, draw=secondarygreen, fill=secondarygreen!20},
    decision/.style={diamond, minimum width=2cm, minimum height=1cm, text centered, draw=warningorange, fill=warningorange!20},
    end/.style={rectangle, rounded corners, minimum width=3cm, minimum height=1cm, text centered, draw=dangerred, fill=dangerred!20},
    arrow/.style={thick,->,>=stealth}
]

\node (start) [start] {Browse Event\\di App};
\node (register) [process, below of=start] {Klik "Join Event"};
\node (check1) [decision, below of=register] {Credit Score\\Cukup?};
\node (waiting) [process, below left of=check1, xshift=-1cm] {Masuk\\Waiting List};
\node (confirmed) [process, below right of=check1, xshift=1cm] {Terkonfirmasi\\Join};
\node (checkin) [process, below of=confirmed] {Check-in di\\Lokasi Event};
\node (matchmaking) [process, below of=checkin] {Sistem Cari\\Lawan};
\node (courtassign) [process, below of=matchmaking] {Dapet Nomor\\Court};
\node (match) [process, below of=courtassign] {Main\\Badminton!};
\node (rating) [process, below of=match] {Kasih Rating\\ke Lawan};
\node (credit) [process, below of=rating] {Update Credit\\Score};
\node (mmr) [end, below of=credit] {Update MMR\\& History};

\draw [arrow] (start) -- (register);
\draw [arrow] (register) -- (check1);
\draw [arrow] (check1) -- node[anchor=south east] {< 40} (waiting);
\draw [arrow] (check1) -- node[anchor=south west] {≥ 40} (confirmed);
\draw [arrow] (waiting) -- (confirmed);
\draw [arrow] (confirmed) -- (checkin);
\draw [arrow] (checkin) -- (matchmaking);
\draw [arrow] (matchmaking) -- (courtassign);
\draw [arrow] (courtassign) -- (match);
\draw [arrow] (match) -- (rating);
\draw [arrow] (rating) -- (credit);
\draw [arrow] (credit) -- (mmr);

\end{tikzpicture}
\caption{Perjalanan Player dari Daftar sampai Selesai Main}
\end{figure}

\subsection{Step 1: Join Event dan Pengecekan Credit Score}

Pas player mau join event, sistem akan cek dulu credit score-nya. Ini kode yang digunakan:

\begin{lstlisting}[language=PHP, caption=Pengecekan Join Event di CreditScoreService.php]
public function canJoinEvent(User $user, Event $event)
{
    $restrictions = $this->checkCreditScoreRestrictions($user);
    
    // Cek credit score basic
    if (!$restrictions['can_join_events']) {
        return [
            'can_join' => false,
            'reason' => 'Credit score terlalu rendah untuk join event.'
        ];
    }

    // Cek untuk premium event
    if ($event->is_premium && !$restrictions['can_join_premium_events']) {
        return [
            'can_join' => false,
            'reason' => 'Credit score tidak mencukupi untuk premium event.'
        ];
    }

    // Cek batas event per minggu
    $weeklyJoins = $this->getUserWeeklyEventJoins($user);
    if ($weeklyJoins >= $restrictions['max_events_per_week']) {
        return [
            'can_join' => false,
            'reason' => 'Telah mencapai batas maksimal event per minggu.'
        ];
    }

    return ['can_join' => true, 'reason' => null];
}
\end{lstlisting}

\subsubsection{Aturan Credit Score}
Ini dia aturan main credit score di SportPWA:

\begin{table}[H]
\centering
\begin{tabular}{|c|c|c|c|}
\hline
\textbf{Credit Score} & \textbf{Bisa Join Event?} & \textbf{Premium Event?} & \textbf{Max Event/Minggu} \\
\hline
0-39 & ❌ Nggak Bisa & ❌ & 0 \\
40-59 & ✅ Bisa & ❌ & 3 \\
60-79 & ✅ Bisa & ✅ Bisa & 10 \\
80-100 & ✅ Bisa & ✅ Bisa & Unlimited \\
\hline
\end{tabular}
\caption{Aturan Credit Score yang Mudah Dipahami}
\end{table}

\subsection{Step 2: Check-in dan Konfirmasi Kehadiran}

Setelah join, player harus check-in di lokasi event. Ada 2 cara:
\begin{itemize}
    \item \textbf{QR Code}: Host scan QR code player
    \item \textbf{Manual}: Host confirm manual di app
\end{itemize}

\section{Sistem Matchmaking yang Fair}

\subsection{Gimana Caranya Sistem Cari Lawan yang Cocok?}

Sistem kita punya algoritma khusus buat carikan lawan yang seimbang. Ini dia cara kerjanya:

\begin{lstlisting}[language=PHP, caption=Algoritma Fair Matchmaking di MatchmakingService.php]
public function createFairMatches(Event $event)
{
    try {
        DB::beginTransaction();

        // Ambil semua peserta yang udah check-in
        $participants = $this->getEligibleParticipants($event);
        
        if ($participants->count() < 2) {
            return [
                'success' => false,
                'message' => 'Minimal 2 orang buat bisa main'
            ];
        }

        // Hitung compatibility score semua pasangan
        $possibleMatches = $this->calculateCompatibilityScores($participants, $event);
        
        // Pilih pasangan terbaik
        $matches = $this->optimizeMatches($possibleMatches, $event);
        
        // Simpan ke database
        $savedMatches = $this->saveMatches($matches, $event);
        
        DB::commit();
        
        return [
            'success' => true,
            'matches' => $savedMatches,
            'total_matches' => count($savedMatches),
        ];

    } catch (\Exception $e) {
        DB::rollBack();
        return ['success' => false, 'message' => 'Ada error nih'];
    }
}
\end{lstlisting}

\subsection{Faktor-faktor yang Dipertimbangkan}

Sistem kita lihat 4 hal buat tentuin pasangan yang cocok:

\begin{enumerate}
    \item \textbf{MMR (40\%)}: Rating skill pemain - yang paling penting
    \item \textbf{Level (25\%)}: Beginner, intermediate, advanced, expert
    \item \textbf{Win Rate (20\%)}: Persentase menang pemain
    \item \textbf{Waiting Time (15\%)}: Berapa lama udah nungguin
\end{enumerate}

\begin{lstlisting}[language=PHP, caption=Calculation Compatibility Score]
private function calculateCompatibility($participant1, $participant2, Event $event)
{
    $player1 = $participant1->user;
    $player2 = $participant2->user;
    
    // Ambil rating untuk olahraga ini
    $rating1 = $this->getUserSportRating($player1, $event->sport_id);
    $rating2 = $this->getUserSportRating($player2, $event->sport_id);

    // Hitung masing-masing faktor
    $mmrScore = $this->calculateMMRCompatibility($rating1['mmr'], $rating2['mmr']);
    $winRateScore = $this->calculateWinRateCompatibility($rating1['win_rate'], $rating2['win_rate']);
    $waitingTimeScore = $this->calculateWaitingTimeBonus($participant1, $participant2);
    $levelScore = $this->calculateLevelCompatibility($rating1['level'], $rating2['level']);

    // Bobot masing-masing faktor
    $weights = [
        'mmr' => 0.4,         // 40% - Paling penting
        'level' => 0.25,      // 25% - Level skill
        'win_rate' => 0.2,    // 20% - Konsistensi
        'waiting_time' => 0.15 // 15% - Fairness buat yang nunggu
    ];

    $totalScore = (
        $mmrScore * $weights['mmr'] +
        $levelScore * $weights['level'] +
        $winRateScore * $weights['win_rate'] +
        $waitingTimeScore * $weights['waiting_time']
    );

    return ['total' => round($totalScore, 2), 'details' => [...]];
}
\end{lstlisting}

\subsection{MMR Compatibility - Semakin Dekat Semakin Bagus}

Ini cara sistem ngitung MMR compatibility:

\begin{lstlisting}[language=PHP, caption=MMR Compatibility Calculation]
private function calculateMMRCompatibility($mmr1, $mmr2)
{
    $difference = abs($mmr1 - $mmr2);
    
    // Beda MMR yang ideal: 0-100 poin
    if ($difference <= 50) {
        return 100; // Perfect match!
    } elseif ($difference <= 100) {
        return 90 - ($difference - 50); // 90-40 range
    } elseif ($difference <= 200) {
        return 40 - (($difference - 100) * 0.3); // 40-10 range
    } else {
        return max(0, 10 - (($difference - 200) * 0.05)); // 10-0 range
    }
}
\end{lstlisting}

\subsubsection{Level Skill dan MMR Range}

\begin{table}[H]
\centering
\begin{tabular}{|c|c|c|}
\hline
\textbf{MMR Range} & \textbf{Level} & \textbf{Deskripsi} \\
\hline
< 800 & Beginner & Baru belajar, masih basic \\
800-1199 & Intermediate & Udah lumayan, paham teknik dasar \\
1200-1599 & Advanced & Jago, teknik dan strategi bagus \\
1600-1999 & Expert & Ahli, level kompetitif \\
≥ 2000 & Professional & Pro, level tournament \\
\hline
\end{tabular}
\caption{Level Skill berdasarkan MMR}
\end{table}

\section{Credit Score System - Sistem Poin Perilaku}

\subsection{Apa itu Credit Score?}

Credit Score itu kayak poin perilaku di SportPWA. Range-nya 0-100:
\begin{itemize}
    \item \textbf{100}: User baru, perilaku perfect
    \item \textbf{80-99}: Bagus banget, nggak ada masalah
    \item \textbf{60-79}: Cukup baik, bisa akses premium
    \item \textbf{40-59}: Agak bermasalah, terbatas aksesnya
    \item \textbf{0-39}: Bermasalah, nggak bisa join event
\end{itemize}

\subsection{Cara Dapet dan Kehilangan Poin}

\subsubsection{Penalty (Minus Poin)}

\begin{table}[H]
\centering
\begin{tabular}{|c|c|c|}
\hline
\textbf{Pelanggaran} & \textbf{Minus Poin} & \textbf{Kapan} \\
\hline
Cancel ≥ 24 jam & -5 & Masih aman \\
Cancel 12-24 jam & -10 & Mulai kena penalty \\
Cancel 6-12 jam & -15 & Lumayan sakit \\
Cancel 2-6 jam & -20 & Sakit banget \\
Cancel < 2 jam & -25 & Maksimal penalty \\
No-Show (nggak dateng) & -30 & Paling parah \\
Rating jelek (< 2.0) & -5 & Dari rating lawan \\
\hline
\end{tabular}
\caption{Penalty Credit Score}
\end{table}

\begin{lstlisting}[language=PHP, caption=Penalty Cancellation di CreditScoreService.php]
private function calculateCancellationPenalty(Event $event)
{
    $hoursBeforeEvent = $this->getHoursBeforeEvent($event);

    if ($hoursBeforeEvent >= 24) {
        return 5;  // -5 poin untuk 24+ jam sebelumnya
    } elseif ($hoursBeforeEvent >= 12) {
        return 10; // -10 poin untuk 12-24 jam sebelumnya
    } elseif ($hoursBeforeEvent >= 6) {
        return 15; // -15 poin untuk 6-12 jam sebelumnya
    } elseif ($hoursBeforeEvent >= 2) {
        return 20; // -20 poin untuk 2-6 jam sebelumnya
    } else {
        return 25; // -25 poin untuk < 2 jam sebelumnya
    }
}
\end{lstlisting}

\subsubsection{Bonus (Plus Poin)}

\begin{table}[H]
\centering
\begin{tabular}{|c|c|c|}
\hline
\textbf{Achievement} & \textbf{Plus Poin} & \textbf{Syarat} \\
\hline
Selesaikan Event & +2 & Check-in dan ikut sampai selesai \\
Rating Bagus & +1 & Dapet rating ≥ 4.0/5.0 \\
5 Event Berturut & +5 & 5 event selesai dalam 30 hari \\
Perfect Attendance & +3 & Nggak cancel dalam 30 hari \\
\hline
\end{tabular}
\caption{Bonus Credit Score}
\end{table}

\begin{lstlisting}[language=PHP, caption=Bonus Event Completion]
public function processEventCompletionBonus(User $user, Event $event)
{
    // Cek udah pernah dapet bonus atau belum
    $existingBonus = CreditScoreLog::where([
        'user_id' => $user->id,
        'event_id' => $event->id,
        'type' => 'event_completion_bonus'
    ])->first();

    if ($existingBonus) {
        return 0; // Udah pernah dapet
    }

    $baseBonus = 2; // Bonus dasar completion
    $this->applyCreditScoreChange($user, $baseBonus, 'event_completion_bonus', 
        "Bonus completion event: {$event->title}", $event->id);

    // Cek bonus berturut-turut
    $consecutiveBonus = $this->checkAndApplyConsecutiveBonus($user);
    
    return $baseBonus + $consecutiveBonus;
}
\end{lstlisting}

\section{MMR System - Rating Skill Pemain}

\subsection{Apa itu MMR?}

MMR (Match Making Rating) adalah angka yang nunjukkin skill level pemain untuk masing-masing olahraga. Sistem kita pakai modifikasi dari Elo Rating yang dipake di catur dan game online.

\subsubsection{Rumus Dasar MMR}

\begin{equation}
MMR_{baru} = MMR_{lama} + K \times (HasilAsli - HasilPrediksi)
\end{equation}

Dimana:
\begin{itemize}
    \item \textbf{K}: Faktor development (10-40)
    \item \textbf{HasilAsli}: 1 (menang), 0.5 (draw), 0 (kalah)
    \item \textbf{HasilPrediksi}: Probabilitas menang berdasarkan MMR
\end{itemize}

\subsection{Implementasi MMR Update}

\begin{lstlisting}[language=PHP, caption=Update MMR After Match di MatchmakingService.php]
private function updateMMRAfterMatch($player1, $player2, $matchResult, $sport)
{
    $rating1 = $this->getUserSportRating($player1, $sport);
    $rating2 = $this->getUserSportRating($player2, $sport);

    // Hitung expected score (prediksi)
    $expected1 = 1 / (1 + pow(10, ($rating2['mmr'] - $rating1['mmr']) / 400));
    $expected2 = 1 - $expected1;

    // Tentukan actual score (hasil asli)
    $actual1 = ($matchResult == 'player1_win') ? 1 : 
               (($matchResult == 'draw') ? 0.5 : 0);
    $actual2 = 1 - $actual1;

    // Hitung K-factor
    $k1 = $this->calculateKFactor($rating1);
    $k2 = $this->calculateKFactor($rating2);

    // Update MMR
    $newMMR1 = $rating1['mmr'] + $k1 * ($actual1 - $expected1);
    $newMMR2 = $rating2['mmr'] + $k2 * ($actual2 - $expected2);

    // Simpan ke database
    $this->updateUserSportRating($player1, $newMMR1, $matchResult);
    $this->updateUserSportRating($player2, $newMMR2, $matchResult);
}
\end{lstlisting}

\subsubsection{K-Factor - Seberapa Cepat MMR Berubah}

\begin{lstlisting}[language=PHP, caption=Calculation K-Factor]
private function calculateKFactor($rating)
{
    $matches = $rating['matches_played'];
    $mmr = $rating['mmr'];
    
    // Player baru: perubahan cepat
    if ($matches < 10) {
        return 40;
    }
    // Player developing: masih cepat
    elseif ($matches < 30) {
        return 30;
    }
    // Berdasarkan skill level
    elseif ($mmr < 1200) {
        return 20; // Lower skill: perubahan sedang
    }
    elseif ($mmr < 1800) {
        return 15; // Intermediate: perubahan lambat
    }
    else {
        return 10; // High skill: perubahan sangat lambat
    }
}
\end{lstlisting}

\section{Court Management - Atur Lapangan dan Antrian}

\subsection{Penjelasan Rumus-Rumus Penting}

Sebelum lanjut ke court management, mari kita bahas rumus-rumus penting yang dipakai sistem dengan penjelasan yang mudah dipahami:

\subsubsection{1. Rumus MMR (Match Making Rating)}

\textbf{Rumus Dasar:}
\begin{equation}
MMR_{baru} = MMR_{lama} + K \times (HasilAsli - HasilPrediksi)
\end{equation}

\textbf{Penjelasan Sederhana:}
\begin{itemize}
    \item \textbf{MMR\_lama}: Rating skill kamu sebelum main (contoh: 1200)
    \item \textbf{K}: Seberapa cepat rating berubah (10-40, makin baru makin cepat)
    \item \textbf{HasilAsli}: Hasil match nyata (Menang=1, Draw=0.5, Kalah=0)
    \item \textbf{HasilPrediksi}: Berapa kemungkinan kamu menang (0-1)
\end{itemize}

\textbf{Contoh Praktis:}
\begin{itemize}
    \item John MMR: 1200, Jane MMR: 1100
    \item Prediksi John menang: 0.64 (64\%)
    \item Hasil: John menang (HasilAsli = 1)
    \item K-factor John: 20 (pemain intermediate)
    \item MMR baru John: 1200 + 20 × (1 - 0.64) = 1200 + 7.2 = 1207
\end{itemize}

\subsubsection{2. Rumus Expected Score (Prediksi Menang)}

\textbf{Rumus:}
\begin{equation}
ExpectedScore = \frac{1}{1 + 10^{(MMR_{lawan} - MMR_{kamu})/400}}
\end{equation}

\textbf{Penjelasan Sederhana:}
Rumus ini ngitung berapa persen kemungkinan kamu menang berdasarkan beda MMR.

\textbf{Contoh Praktis:}
\begin{itemize}
    \item Kamu: MMR 1200, Lawan: MMR 1100
    \item Beda MMR: 1200 - 1100 = 100 (kamu lebih tinggi)
    \item Expected Score = 1/(1 + 10^(-100/400)) = 1/(1 + 0.56) = 0.64
    \item Artinya: 64\% kemungkinan kamu menang
\end{itemize}

\subsubsection{3. Rumus K-Factor (Seberapa Cepat Rating Berubah)}

\textbf{Logika K-Factor:}
\begin{itemize}
    \item \textbf{Player Baru (< 10 match)}: K = 40 (perubahan cepat)
    \item \textbf{Player Developing (< 30 match)}: K = 30 (masih cepat)
    \item \textbf{Low Skill (MMR < 1200)}: K = 20 (sedang)
    \item \textbf{Intermediate (MMR 1200-1800)}: K = 15 (lambat)
    \item \textbf{High Skill (MMR > 1800)}: K = 10 (sangat lambat)
\end{itemize}

\textbf{Kenapa Begini?}
\begin{itemize}
    \item Player baru: Sistemnya belum tau skill asli, jadi perubahan cepat
    \item Player expert: Skillnya udah stabil, jadi perubahan lambat
    \item Skill rendah: Masih banyak ruang improve, jadi perubahan sedang
\end{itemize}

\subsubsection{4. Rumus Compatibility Score (Kecocokan Pasangan)}

\textbf{Rumus:}
\begin{equation}
CompatibilityScore = MMR_{score} \times 0.4 + Level_{score} \times 0.25 + WinRate_{score} \times 0.2 + Wait_{score} \times 0.15
\end{equation}

\textbf{Contoh Perhitungan:}
\begin{itemize}
    \item \textbf{MMR Score}: Beda MMR 80 poin → Score 90 (40\% × 90 = 36)
    \item \textbf{Level Score}: Same level → Score 100 (25\% × 100 = 25)
    \item \textbf{Win Rate Score}: Beda win rate 15\% → Score 75 (20\% × 75 = 15)
    \item \textbf{Waiting Score}: Rata-rata nunggu 45 menit → Score 85 (15\% × 85 = 12.75)
    \item \textbf{Total}: 36 + 25 + 15 + 12.75 = 88.75 (Very Good Match!)
\end{itemize}

\subsubsection{5. Rumus MMR Compatibility}

\textbf{Logika Sederhana:}
\begin{lstlisting}[language=text, caption=MMR Compatibility Logic]
Beda MMR 0-50 poin   → Score 100 (Perfect!)
Beda MMR 51-100 poin → Score 90-40 (Bagus)
Beda MMR 101-200 poin → Score 40-10 (Lumayan)
Beda MMR >200 poin   → Score 10-0 (Jelek)
\end{lstlisting}

\textbf{Contoh:}
\begin{itemize}
    \item Player A: MMR 1200, Player B: MMR 1250
    \item Beda MMR: |1200 - 1250| = 50 poin
    \item Compatibility Score: 100 (Perfect match!)
\end{itemize}

\subsubsection{6. Rumus Waiting Time Bonus}

\textbf{Logika:}
Semakin lama nunggu, semakin tinggi prioritas untuk dicarikan lawan.

\begin{lstlisting}[language=text, caption=Waiting Time Bonus Logic]
Nunggu ≥ 60 menit → Score 100 (Prioritas tertinggi)
Nunggu 30-60 menit → Score 80-100 (Prioritas tinggi)
Nunggu 15-30 menit → Score 60-80 (Prioritas sedang)
Nunggu < 15 menit → Score 40-60 (Prioritas rendah)
\end{lstlisting}

\subsubsection{7. Rumus Credit Score Penalty}

\textbf{Logika Cancellation:}
\begin{lstlisting}[language=text, caption=Cancellation Penalty Logic]
Cancel ≥ 24 jam sebelumnya → -5 poin (Masih wajar)
Cancel 12-24 jam → -10 poin (Mulai kena)
Cancel 6-12 jam → -15 poin (Lumayan sakit)
Cancel 2-6 jam → -20 poin (Sakit banget)
Cancel < 2 jam → -25 poin (Maksimal penalty)
\end{lstlisting}

\textbf{Contoh Praktis:}
\begin{itemize}
    \item Event badminton jam 19:00
    \item Kamu cancel jam 17:30 (1.5 jam sebelumnya)
    \item Credit score kamu: 85 → 85 - 25 = 60
    \item Sekarang kamu nggak bisa join premium event lagi sampai credit score naik
\end{itemize}

\subsection{Tips Memahami Rumus-Rumus Ini}

\begin{itemize}
    \item \textbf{MMR}: Makin tinggi, makin jago. Perubahan bergantung pengalaman.
    \item \textbf{Compatibility}: Makin tinggi score, makin cocok sebagai lawan.
    \item \textbf{Credit Score}: Makin tinggi, makin bebas akses fitur premium.
    \item \textbf{Waiting Time}: Makin lama nunggu, makin prioritas dapat lawan.
    \item \textbf{K-Factor}: Menentukan seberapa cepat MMR berubah.
\end{itemize}

\subsection{Skenario Praktis: Step-by-Step Calculation}

Mari kita lihat contoh nyata gimana rumus-rumus ini bekerja dalam satu skenario:

\subsubsection{Skenario: Event Badminton Sore}

\textbf{Setting:}
\begin{itemize}
    \item Event badminton jam 19:00
    \item 4 pemain check-in: Alex, Budi, Citra, Doni
    \item Sistem harus carikan 2 pasangan yang fair
\end{itemize}

\textbf{Data Pemain:}
\begin{table}[H]
\centering
\begin{tabular}{|c|c|c|c|c|c|}
\hline
\textbf{Nama} & \textbf{MMR} & \textbf{Matches} & \textbf{Win Rate} & \textbf{Credit Score} & \textbf{Check-in} \\
\hline
Alex & 1250 & 25 & 65\% & 85 & 18:45 \\
Budi & 1180 & 15 & 60\% & 90 & 18:50 \\
Citra & 1200 & 30 & 70\% & 75 & 18:40 \\
Doni & 1320 & 40 & 75\% & 95 & 18:55 \\
\hline
\end{tabular}
\caption{Data Pemain untuk Matchmaking}
\end{table}

\textbf{Step 1: Hitung Compatibility Scores}

\textit{Pasangan 1: Alex vs Budi}
\begin{itemize}
    \item MMR Diff: |1250 - 1180| = 70 → MMR Score = 85
    \item Level: Advanced vs Intermediate → Level Score = 75
    \item Win Rate Diff: |65 - 60| = 5\% → Win Rate Score = 95
    \item Avg Wait: (15+10)/2 = 12.5 menit → Wait Score = 50
    \item \textbf{Total}: 85×0.4 + 75×0.25 + 95×0.2 + 50×0.15 = \textbf{79.25}
\end{itemize}

\textit{Pasangan 2: Alex vs Citra}
\begin{itemize}
    \item MMR Diff: |1250 - 1200| = 50 → MMR Score = 100
    \item Level: Advanced vs Advanced → Level Score = 100
    \item Win Rate Diff: |65 - 70| = 5\% → Win Rate Score = 95
    \item Avg Wait: (15+20)/2 = 17.5 menit → Wait Score = 55
    \item \textbf{Total}: 100×0.4 + 100×0.25 + 95×0.2 + 55×0.15 = \textbf{93.25}
\end{itemize}

\textit{Pasangan 3: Alex vs Doni}
\begin{itemize}
    \item MMR Diff: |1250 - 1320| = 70 → MMR Score = 85
    \item Level: Advanced vs Expert → Level Score = 75
    \item Win Rate Diff: |65 - 75| = 10\% → Win Rate Score = 80
    \item Avg Wait: (15+5)/2 = 10 menit → Wait Score = 45
    \item \textbf{Total}: 85×0.4 + 75×0.25 + 80×0.2 + 45×0.15 = \textbf{77.5}
\end{itemize}

\textbf{Step 2: Pilih Pasangan Terbaik}

Ranking berdasarkan compatibility score:
\begin{enumerate}
    \item \textbf{Alex vs Citra: 93.25} (TERPILIH!)
    \item Budi vs Doni: 82.0 (TERPILIH!)
    \item Alex vs Budi: 79.25
    \item Alex vs Doni: 77.5
\end{enumerate}

\textbf{Step 3: Simulasi Hasil Match}

\textit{Match 1: Alex (MMR 1250) vs Citra (MMR 1200)}
\begin{itemize}
    \item Expected Score Alex: 1/(1+10^{(1200-1250)/400}) = 0.57 (57\%)
    \item Expected Score Citra: 1 - 0.57 = 0.43 (43\%)
    \item Hasil: Alex menang
    \item K-factor Alex: 15 (intermediate, 25 matches)
    \item K-factor Citra: 15 (intermediate, 30 matches)
    \item MMR baru Alex: 1250 + 15×(1-0.57) = 1250 + 6.45 = \textbf{1256}
    \item MMR baru Citra: 1200 + 15×(0-0.43) = 1200 - 6.45 = \textbf{1194}
\end{itemize}

\textbf{Step 4: Update Credit Score}

Karena semua pemain datang dan main sampai selesai:
\begin{itemize}
    \item Alex: 85 + 2 (completion bonus) = \textbf{87}
    \item Budi: 90 + 2 = \textbf{92}
    \item Citra: 75 + 2 = \textbf{77}
    \item Doni: 95 + 2 = \textbf{97}
\end{itemize}

\subsubsection{Analisis Hasil}

\textbf{Kenapa Alex vs Citra Dipilih?}
\begin{itemize}
    \item MMR sangat dekat (50 poin) → Perfect compatibility
    \item Same skill level (Advanced)
    \item Win rate hampir sama
    \item Citra udah nunggu lebih lama
\end{itemize}

\textbf{Pembelajaran dari Skenario:}
\begin{itemize}
    \item Sistem prioritaskan keseimbangan MMR dan level
    \item Waiting time kasih bonus ke yang udah lama nunggu
    \item Perubahan MMR proporsional dengan expected vs actual result
    \item Credit score naik untuk semua yang complete event
\end{itemize}

\subsection{Sistem Antrian yang Fair}

Sistem antrian di SportPWA diatur berdasarkan beberapa faktor:

\begin{enumerate}
    \item \textbf{Waiting Time}: Yang udah lama nunggu dapat prioritas
    \item \textbf{Premium Status}: Member premium dapat sedikit prioritas
    \item \textbf{MMR Balance}: Sistem coba buat match yang seimbang
    \item \textbf{Credit Score}: Score tinggi dapat preferensi
\end{enumerate}

\begin{lstlisting}[language=PHP, caption=Get Queue Info di MatchmakingService.php]
public function getQueueInfo(Event $event)
{
    $participants = $this->getEligibleParticipants($event);
    $activeMatches = MatchHistory::where('event_id', $event->id)
        ->whereIn('match_status', ['scheduled', 'ongoing'])
        ->count();

    $waitingPlayers = $participants->filter(function($participant) {
        return !$participant->user->hasActiveMatch();
    });

    return [
        'total_participants' => $participants->count(),
        'active_matches' => $activeMatches,
        'waiting_players' => $waitingPlayers->count(),
        'can_create_matches' => $waitingPlayers->count() >= 2,
        'queue_list' => $waitingPlayers->map(function($participant) {
            return [
                'user' => $participant->user,
                'waiting_since' => $participant->confirmed_at ?? $participant->created_at,
                'waiting_minutes' => Carbon::now()->diffInMinutes($participant->confirmed_at ?? $participant->created_at)
            ];
        })->sortByDesc('waiting_minutes')->values()
    ];
}
\end{lstlisting}

\subsection{Host Management Tools}

Host punya beberapa tools untuk mengatur court:

\begin{lstlisting}[language=PHP, caption=Court Assignment di MatchmakingController.php]
public function assignCourt(Request $request, $eventId)
{
    $validator = Validator::make($request->all(), [
        'court_number' => 'required|integer|min:1|max:20',
        'player1_id' => 'required|exists:users,id',
        'player2_id' => 'required|exists:users,id|different:player1_id',
    ]);

    $event = Event::findOrFail($eventId);
    $user = Auth::user();

    // Cuma host yang bisa assign court
    if ($event->host_id !== $user->id && !$user->hasRole('admin')) {
        return response()->json([
            'status' => 'error',
            'message' => 'Hanya host yang dapat mengatur court.'
        ], 403);
    }

    $result = $this->courtManagementService->assignCourt(
        $event,
        $request->court_number,
        $request->player1_id,
        $request->player2_id,
        $user->id
    );

    return response()->json([
        'status' => $result['success'] ? 'success' : 'error',
        'message' => $result['message'],
        'data' => $result['success'] ? ['match' => $result['match']] : null
    ]);
}
\end{lstlisting}

\section{API Integration - Cara Pakai dari Frontend}

\subsection{Key API Endpoints yang Penting}

Ini beberapa API endpoint utama yang dipakai frontend:

\begin{lstlisting}[language=bash, caption=API Endpoints Utama]
# Event Management
POST /api/events/{eventId}/join          # Join event
DELETE /api/events/{eventId}/leave       # Leave event
POST /api/events/{eventId}/check-in/{participantId} # Check-in

# Matchmaking
POST /api/matchmaking/{eventId}/fair-matches # Buat fair matches
GET /api/matchmaking/{eventId}/status    # Status matchmaking

# Court Management
GET /api/matchmaking/{eventId}/court-status # Status court
POST /api/matchmaking/{eventId}/assign-court # Assign court
POST /api/matchmaking/{eventId}/override-player # Override player

# Credit Score
GET /api/credit-score                    # Liat credit score
GET /api/credit-score/preview-cancellation/{eventId} # Preview penalty
\end{lstlisting}

\subsection{Contoh Response API}

\begin{lstlisting}[language=JSON, caption=Response Fair Matchmaking API]
{
    "status": "success",
    "message": "Fair matchmaking berhasil dibuat!",
    "data": {
        "event_id": 123,
        "event_title": "Badminton Sore Hari",
        "algorithm_used": "Fair Matchmaking Algorithm v2.0",
        "total_matches": 3,
        "matched_players": 6,
        "waiting_players": 2,
        "matches": [
            {
                "id": 456,
                "player1": {
                    "id": 1,
                    "name": "John Doe",
                    "mmr": 1200
                },
                "player2": {
                    "id": 2,
                    "name": "Jane Smith", 
                    "mmr": 1180
                },
                "compatibility_score": {
                    "total": 92.5,
                    "details": {
                        "mmr_score": 95,
                        "level_score": 100,
                        "win_rate_score": 85,
                        "waiting_time_score": 80
                    }
                },
                "court_number": 1,
                "status": "scheduled"
            }
        ]
    }
}
\end{lstlisting}

\section{Tips dan Best Practices}

\subsection{Untuk Players}

\begin{itemize}
    \item \textbf{Jaga Credit Score}: Usahakan di atas 60 buat akses full
    \item \textbf{Cancel dengan Bijak}: Cancel minimal 24 jam sebelumnya
    \item \textbf{Datang Tepat Waktu}: Check-in sesuai jadwal
    \item \textbf{Kasih Rating Fair}: Rating yang jujur membantu sistem
    \item \textbf{Bermain Sportif}: Good sportsmanship = rating bagus
\end{itemize}

\subsection{Untuk Hosts}

\begin{itemize}
    \item \textbf{Gunakan Fair Matchmaking}: Biar match lebih seimbang
    \item \textbf{Override Seperlunya}: Jangan asal override premium users
    \item \textbf{Manage Queue}: Perhatikan yang udah lama nunggu
    \item \textbf{Check-in Tertib}: Pastikan semua peserta check-in
    \item \textbf{Rating System}: Ingatkan players untuk kasih rating
\end{itemize}

\subsection{Untuk Developers}

\begin{itemize}
    \item \textbf{Database Transactions}: Selalu pakai DB transaction untuk matchmaking
    \item \textbf{Error Handling}: Comprehensive error handling di semua service
    \item \textbf{Logging}: Log semua aktivitas penting untuk debugging
    \item \textbf{Caching}: Cache hasil compatibility calculation
    \item \textbf{Real-time Updates}: Pakai websockets untuk update live
\end{itemize}

\section{Troubleshooting - Masalah yang Sering Muncul}

\subsection{Masalah Matchmaking}

\textbf{Q: Kenapa nggak bisa buat match?}\\
A: Cek minimal 2 peserta yang eligible dan nggak lagi main

\textbf{Q: Match nggak balance?}\\
A: Adjust skill tolerance di parameter matchmaking

\textbf{Q: Premium user kena override terus?}\\
A: Premium user harusnya protected dari arbitrary switching

\subsection{Masalah Credit Score}

\textbf{Q: Credit score turun terus?}\\
A: Cek history penalty, mungkin sering cancel atau no-show

\textbf{Q: Nggak bisa join event?}\\
A: Credit score harus minimal 40 untuk join event

\textbf{Q: Bonus nggak masuk?}\\
A: Cek udah check-in dan complete event atau belum

\subsection{Masalah MMR}

\textbf{Q: MMR nggak update setelah menang?}\\
A: Cek match sudah di-record sebagai completed

\textbf{Q: MMR turun banyak setelah kalah?}\\
A: Normal jika lawan MMR-nya jauh lebih rendah

\section{Kesimpulan}

SportPWA Matchmaking System adalah solusi lengkap untuk:

\begin{enumerate}
    \item \textbf{Fair Play}: Algoritma yang smart buat match seimbang
    \item \textbf{Behavior Management}: Credit score yang mendorong perilaku baik
    \item \textbf{Skill Tracking}: MMR system yang akurat dan fair
    \item \textbf{User Experience}: Flow yang smooth dari join sampai finish
    \item \textbf{Scalability}: Architecture yang bisa handle growth
\end{enumerate}

Sistem ini udah terintegrasi dengan baik di Laravel backend dan React frontend, memberikan pengalaman yang smooth untuk semua user - dari player casual sampai tournament organizer.

\vspace{1cm}

\textit{Dokumen ini akan terus diupdate seiring dengan perkembangan SportPWA. Happy playing! 🏸🎾⚽}

\end{document}